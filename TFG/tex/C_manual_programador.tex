\apendice{Manual del desarrollador / programador / investigador.} % usar el término que mejor se corresponda.

\section{Estructura de directorios}

Todos los archivos de este proyecto se pueden encontrar en el \href{https://github.com/diegotrascasa/TFG_Diego_Trascasa_Garcia}{repositorio de GitHub}. A continuación, se describe el contenido de cada archivo para ayudar a comprender y analizar el trabajo.

\begin{itemize}
    \item \textbf{Arduino/}: Carpeta que contiene todo el material relacionado con el desarrollo del software para la placa Arduino.
    \begin{itemize}
        \item \textbf{version 1/}: Scripts iniciales de prueba para los sensores KY039 y AD8232.
        \begin{itemize}
            \item \textbf{KY039\_bluetooth.ino}: Código para probar el sensor KY039 con el módulo Bluetooth HC-05, verificando la transmisión de datos de ECG por Bluetooth.
            \item \textbf{KY039\_usb.ino}: Código para probar el sensor KY039 con conexión USB, evaluando la transmisión de datos directamente al ordenador.
            \item \textbf{AD82CONPYTHONYELECTRODOS\_bluetooth.ino}: Código para probar el sensor AD8232 con el módulo Bluetooth HC-05, comprobando la calidad y consistencia de los datos transmitidos.
            \item \textbf{AD82CONPYTHONYELECTRODOS\_usb.ino}: Código para probar el sensor AD8232 con conexión USB, optimizando la recogida de datos de ECG y su envío a la aplicación.
        \end{itemize}
        \item \textbf{version 2/}: Versión final del código de Arduino utilizado en el proyecto.
        \begin{itemize}
            \item \textbf{AD8232\_usb\_y\_led.ino}: Código mejorado para el sensor AD8232 con comunicación por USB, incluyendo reducción de ruido, mejora de la calidad de datos, LED sincronizado con el pulso y auto-calibración del umbral para adaptarse a cada paciente.
        \end{itemize}
    \end{itemize}

    \item \textbf{README.md}: Explicación general del proyecto.


    \item \textbf{fotos/}: Imágenes utilizadas en la página web de Streamlit. Esta carpeta contiene todos los recursos visuales que se han utilizado para mejorar la interfaz de usuario de la aplicación web.

    \item \textbf{data\_ecg/}: Contiene un enlace a Google Drive con los archivos de entrenamiento, prueba y datos normales de ECG. Estos datos son fundamentales para el entrenamiento y evaluación del modelo de predicción de latidos cardíacos.

    \item \textbf{streamlit.py}: Código principal de la aplicación web desarrollada con Streamlit. Este script contiene la lógica y el diseño de la interfaz web, permitiendo la interacción del usuario con las distintas funcionalidades del sistema, incluyendo la visualización de datos de ECG en tiempo real y el análisis de datos.

    \item \textbf{serialmonitor.py}: Ventana de escritorio para la visualización en vivo y grabación de datos, desarrollada con Tkinter. Esta aplicación permite la monitorización en tiempo real de las señales de ECG, así como la posibilidad de grabar los datos para su posterior análisis.

    \item \textbf{AnalisisBueno.ipynb}: Notebook que explica el proceso de tratamiento de datos y la elección del modelo de predicción Random Forest. Proporciona un análisis detallado de los métodos y técnicas utilizadas para el preprocesamiento de datos y la selección del modelo de machine learning.

    \item \textbf{TFG}: Link con una carpeta con todo lo necesario para descargar y que el proyecto funcione correctamente siguiendo la instrucciones mencionadas anteriormente.

    \item \textbf{model.py}: Script para el entrenamiento del modelo de predicción que genera el archivo \textit{ecg\_model.pkl}. Este archivo contiene el modelo entrenado que se utiliza para predecir los tipos de latidos cardíacos a partir de los datos de ECG.

    \item \textbf{Diagrama\_Gantt.xlsx}: Diagrama de Gantt en formato Excel que muestra la planificación temporal del proyecto. El diagrama detalla las tareas, hitos y su duración a lo largo del tiempo, y se encuentra disponible para su descarga y revisión en el repositorio.


    \item 
    \textbf{Documentación\_Proyecto/}: Carpeta que contiene todos los apartados de la memoria del proyecto en formato LateX.
    \begin{itemize}
        \item \textbf{img/}: Guarda todas las imágenes que se usan en la memoria y los anexos.
        \item \textbf{tex/}: Contiene los capítulos de la memoria y los diferentes anexos en formato LaTeX.
    \end{itemize}

    \item \textbf{memoria\_Diego\_Trascasa\_García.pdf}: Documento PDF con la memoria completa.
    \item \textbf{anexo\_Diego\_Trascasa\_García.pdf}: Documento PDF con el anexo completo.
    
    \item \textbf{Video\_demostracion}: Archivo con el enlace al video de domostración del funcionamiento del proyecto.
\end{itemize}


\subsection{Instrucciones para la modificación o mejora del proyecto}

Al ser el primer prototipo de este proyecto, hay muchos aspectos de mejora para el futuro que se pueden implementar en futuras versiones, como por ejemplo añadir mas sensores para mejorar la fiabilidad o mejorar la calidad de todas las herramientas que se han incluido en la aplicación web.

En primer lugar, es crucial mejorar la precisión del modelo de predicción. Para ello, es necesario aumentar la calidad y cantidad de los datos de entrenamiento, utilizando bases de datos más extensas y diversificadas. Además, se debe permitir que el modelo pueda clasificar y añadir etiquetas de patologías directamente, no solo clasificar los tipos de latido de cada segmento.

Otro aspecto de mejora es la implementación de una conexión inalámbrica estable. Se podría incorporar un módulo Bluetooth o WiFi para permitir la comunicación inalámbrica entre el dispositivo de monitoreo y la aplicación web, mejorando la comodidad y movilidad del usuario, eliminando la dependencia de la conexión USB.

Además, se podrían incluir elementos como una batería recargable, aumentando la autonomía del sistema. Incluir una alimentación mediante baterías recargables también aumentará la portabilidad del dispositivo.

La ampliación de las funcionalidades de la aplicación web también es una prioridad. Integrar herramientas adicionales para el análisis de los datos de ECG, como la detección automática de diferentes tipos de arritmias, y generar informes detallados para los profesionales de la salud facilitará el diagnóstico y seguimiento del paciente.


Por último, es esencial implementar medidas adicionales para asegurar la privacidad y seguridad de los datos de los pacientes. Cumplir con normativas y estándares internacionales de protección de datos garantizará la confidencialidad de la información médica.

Estas recomendaciones están destinadas a guiar el desarrollo futuro del proyecto, asegurando su evolución y mejorando su utilidad y accesibilidad.





