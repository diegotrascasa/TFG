\apendice{Anexo de sostenibilización curricular}

\section{Introducción}
Este anexo reflexiona sobre los elementos de sostenibilidad integrados en el desarrollo del proyecto de monitoreo de ECG. A lo largo del Trabajo de Fin de Grado, se han abordado distintas dimensiones de la sostenibilidad, especialmente en los ámbitos tecnológico, social y medioambiental, que son cruciales para asegurar que el proyecto contribuya de manera positiva tanto a la sociedad como al entorno.

\subsection{Sostenibilidad tecnológica}
El proyecto ha promovido la sostenibilidad tecnológica a través del uso de tecnologías de bajo consumo y la implementación de sistemas que prolongan la vida útil de los dispositivos electrónicos. La elección del sensor AD8232 para monitorizar el ECG se fundamenta en su eficiencia y precisión, minimizando la generación de desechos electrónicos.

\subsection{Sostenibilidad social}
Socialmente, el proyecto apunta a mejorar la calidad de vida de los pacientes al proporcionar un monitoreo constante y confiable, facilitando la detección precoz de posibles afecciones cardíacas. Esto se alinea con los objetivos de sostenibilidad de mejorar el bienestar humano y ofrecer accesibilidad a tecnologías sanitarias avanzadas.

\subsection{Sostenibilidad medioambiental}
Desde una perspectiva medioambiental, se ha buscado minimizar el impacto ecológico del proyecto eligiendo componentes reutilizables y reciclables. Además, el diseño y desarrollo del dispositivo contemplan la optimización del consumo de energía.

\subsection{Reflexión personal}
Personalmente, este proyecto ha permitido desarrollar una conciencia más profunda sobre cómo la ingeniería puede y debe integrar principios de sostenibilidad. Ha sido esencial considerar cómo las decisiones técnicas no solo afectan el resultado funcional de un dispositivo, sino también su impacto social y en el medioambiente a largo plazo.

Para más información sobre las directrices de sostenibilidad de la CRUE, puede consultar el documento completo en \cite{crue-sostenibilidad}