\capitulo{1}{Introducción}


La detección temprana de irregularidades en la actividad cardíaca es clave para la prevención y el tratamiento de enfermedades cardiovasculares. Con la creciente incidencia de enfermedades cardíacas a nivel mundial, surge la necesidad de herramientas accesibles a toda la población que permitan la supervisión constante del estado del corazón. Según la Organización Mundial de la Salud (OMS), las enfermedades cardiovasculares son la principal causa de muerte en el mundo, siendo responsables de aproximadamente 17.9 millones de muertes al año, lo que representa el 31\% de todas las muertes globales \cite{who-cvd}. Este Trabajo de Fin de Grado (TFG) se centra en el desarrollo de un sistema de monitoreo cardíaco en tiempo real junto a una predicción de cada tipo de ciclo cardiaco que nos alerta de irregularidades en él, se utiliza una combinación de hardware y software para ofrecer una solución económica y eficaz.


El proyecto se basa en la utilización de un sensor de ECG (Electrocardiograma) conectado a un microcontrolador Arduino, que envía los datos a una aplicación web para su visualización, análisis y almacenamiento. La implementación de este sistema busca proporcionar a los usuarios una herramienta que no solo monitorice la actividad cardíaca en tiempo real, sino que también grabe los datos para análisis posteriores, indentificando patrones relevantes en cada ciclo cardiaco y asignando una etiqueta a cada segmento.

El sistema incluye una aplicación de escritorio desarrollada con Tkinter, que permite la conexión y visualización de los datos en tiempo real y grabación en un archivo de tipo xls. Además tiene una interfaz web interactiva creada con Streamlit para el análisis de los datos de ECG. La información se almacena en una base de datos SQLite, permitiendo un acceso rápido y eficiente a los datos históricos.

Este documento detalla el proceso de desarrollo del sistema de monitoreo cardíaco, comenzando con la definición de objetivos y fundamentos teóricos, seguido por una descripción de la metodología y las herramientas empleadas. Se incluyen los resultados obtenidos y las conclusiones del trabajo realizado, así como una discusión sobre posibles mejoras y expansiones futuras del proyecto.

El objetivo principal de este proyecto es contribuir a la innovación en el campo de la salud, ofreciendo una solución tecnológica que facilite el monitoreo continuo de la actividad cardíaca, mejorando así la calidad de vida de los pacientes y proporcionando a los profesionales de la salud una herramienta adicional para el diagnóstico y seguimiento de enfermedades cardíacas. En los anexos se proporciona información adicional que podría ser interesante sobre el desarrollo del proyecto.




